%%%%%%%%%%%%%%%%%%%%%%%%%%%%%%%%%%%%%%%%%%%%%%%%%%%%%%%%%%
%%%%%%%%%%%%%%%%%%%%%%%%%%%%%%%%%%%%%%%%%%%%%%%%%%%%%%%%%%
%%%%%%%%%%%%%%%%%%%%%%%%%%%%%%%%%%%%%%%%%%%%%%%%%%%%%%%%%%
%%%%%%%%%%%%%%%%%%%%%%%%%%%%%%%%%%%%%%%%%%%%%%%%%%%%%%%%%%
\chapter{Burgers}

%%%%%%%%%%%%%%%%%%%%%%%%%%%%%%%%%%%%%%%%%%%%%%%%%%%%%%%%%%
%%%%%%%%%%%%%%%%%%%%%%%%%%%%%%%%%%%%%%%%%%%%%%%%%%%%%%%%%%
%%%%%%%%%%%%%%%%%%%%%%%%%%%%%%%%%%%%%%%%%%%%%%%%%%%%%%%%%%
\section{Physics}

The inviscid Burgers equation was first introduced by Bateman in 1915, and models the behavior of a one-dimensional inviscid incompressible fluid flow \cite{bateman1915}, before being studied by Burgers in 1948 \cite{burgers1948}:

\begin{equation}
\label{eq:burgers}
	\dpart{u}{t} + u \dpart{u}{x} = 0.
\end{equation}

We consider the resolution of the Burgers equation on a domain of length $L$, along with the following initial and boundary conditions:

\begin{equation}
\label{eq:burgers_bc}
\begin{split}
	u(x,0)	&= u_\text{target}, \\
	u(0,t) 	&= u_\text{target} + \mathcal{U} ( -\sigma, \sigma), \\
	\dpart{u}{x} (L,t) &= 0,
\end{split}
\end{equation}

where $\sigma$ is the noise level introduced at the inlet, and $u_\text{target}$ is a constant value. The convection of the random inlet signal leads to a noisy solution in the domain, as is depicted in figure \ref{fig:burgers_free}. The initial perturbations steepen while propagating downstream to eventually form shocks.

%%%%%%%%%%%%
%%%%%%%%%%%%
\begin{figure}
\centering
%%%%%%%%%%%%
%%%%%%%%%%%%
\fbox{\includegraphics[width=.5\textwidth]{fig/burgers/free.png}}
%%%%%%%%%%%%
\caption{\textbf{Uncontrolled solution of the Burgers equation} with uniform random noise excitation at the inlet. The vertical bar indicates the position of the controller.}
\label{fig:burgers_free}
\end{figure} 
%%%%%%%%%%%%
%%%%%%%%%%%%

%%%%%%%%%%%%%%%%%%%%%%%%%%%%%%%%%%%%%%%%%%%%%%%%%%%%%%%%%%
%%%%%%%%%%%%%%%%%%%%%%%%%%%%%%%%%%%%%%%%%%%%%%%%%%%%%%%%%%
%%%%%%%%%%%%%%%%%%%%%%%%%%%%%%%%%%%%%%%%%%%%%%%%%%%%%%%%%%
\section{Discretization}

The Burgers equation (\ref{eq:burgers}) is discretized in time with a finite volume approach. The convective term is discretized using a TVD scheme with a Van Leer flux limiter, while the time marching is performed using a second-order finite-difference scheme.

%%%%%%%%%%%%%%%%%%%%%%%%%%%%%%%%%%%%%%%%%%%%%%%%%%%%%%%%%%
%%%%%%%%%%%%%%%%%%%%%%%%%%%%%%%%%%%%%%%%%%%%%%%%%%%%%%%%%%
%%%%%%%%%%%%%%%%%%%%%%%%%%%%%%%%%%%%%%%%%%%%%%%%%%%%%%%%%%
\section{Environment}

The goal of the environment is to control a pointwise forcing source term $a(t)$ on the right-hand side of (\ref{eq:burgers}), in order to damp the noise transported from the inlet. The forcing is applied at $x_p = 0.3$, while the length of the domain is set to $L=1$. The actions provided to the environment, which are expected to be in $\left[-1, 1\right]$, are then scale by an \textit{ad-hoc} non-dimensional amplitude factor $A=10$. The field is initially set equal to $u_\text{target} = 0.5$, and the variance of the inlet noise is chosen equal to $\sigma = 0.1$. The spatial discretization step is set to $\Delta x = \num{2e-3}$, while the numerical time-step is equal to $\Delta t = \num{4e-3}$ time units. The action duration is set to $\Delta t_\text{act} = 0.05$ time units, for a total episode duration equal to $10$ time units, corresponding to $200$ actions. The observations provided to the agent are the $n_\text{obs} = 20$ values of $u$ upstream of the actuator. Finally, the reward is computed as:

\begin{equation}
\label{eq:burgers_reward}
	r(t) = - \Delta x \norm{u(x,t) - u_\text{target}}_2 \text{ with } x \in \left[ x_p, L \right].
\end{equation}

%%%%%%%%%%%%%%%%%%%%%%%%%%%%%%%%%%%%%%%%%%%%%%%%%%%%%%%%%%
%%%%%%%%%%%%%%%%%%%%%%%%%%%%%%%%%%%%%%%%%%%%%%%%%%%%%%%%%%
%%%%%%%%%%%%%%%%%%%%%%%%%%%%%%%%%%%%%%%%%%%%%%%%%%%%%%%%%%
\section{Results}

The environment as described in the previous section is referred to as \codeinline{burgers-v0}, in the fashion of the \textsc{gym} environments (the default parameters of the environment are provided in table \ref{table:burgers_parameters}). In this section, we provide some results related to its resolution using a \ppo agent (see the general hyperparameters in table \ref{table:default_ppo_parameters}). For the training, we set $n_\text{rollout} = 4000$, $n_\text{batch} = 2$, $n_\text{epoch} = 32$ and $n_\text{max} = 500k$.

The score curve is shown in figure \ref{fig:burgers_score}, while a plot of the time evolutions of the controlled environment is shown in figure \ref{fig:burgers_control}. As can be observed, the agent successfully damps the transported inlet noise following a kind of opposition control strategy.

%%%%%%%%%%%%
%%%%%%%%%%%%
\begin{table}[h!]
    \footnotesize
    \caption{\textbf{Default parameters used for the \codeinline{burgers-v0} environment.}}
    \label{table:burgers_parameters}
    \centering
    \begin{tabular}{rll}
        \toprule
        	\codeinline{L}			& domain length		& $1$\\
        \codeinline{u_target}		& target value			& $0.5$\\
	\codeinline{sigma}		& inlet noise level		& $0.1$\\
	\codeinline{amp}		& control amplitude		& $10$\\
	\codeinline{ctrl_pos}		& control position		& $0.3$\\
        \bottomrule
    \end{tabular}
\end{table}
%%%%%%%%%%%%
%%%%%%%%%%%%
%%
%%%%%%%%%%%%%
%%%%%%%%%%%%
\begin{figure}
\centering
%%%%%%%%%%%%
\begin{tikzpicture}[	trim axis left, trim axis right, font=\scriptsize,
				upper/.style={	name path=upper, smooth, draw=none},
				lower/.style={	name path=lower, smooth, draw=none},]
	\begin{axis}[	xmin=0, xmax=200000, scale=0.75,
				ymin=-4, ymax=0,
				scaled x ticks=false,
				xtick={0, 50000, 100000, 150000, 200000},
				xticklabels={$0$,$50k$,$100k$,$150k$,$200k$},
				legend cell align=left, legend pos=south east,
				legend style={nodes={scale=0.8, transform shape}},
				every tick label/.append style={font=\scriptsize},
				grid=major, xlabel=transitions, ylabel=score]
				
		\legend{no control, 1 jet, 5 jets, 10 jets}
		
		\addplot[thick, opacity=0.7, dash pattern=on 2pt]	coordinates {(0,-2.90) (200000,-2.90)};
		
		\addplot [upper, forget plot] 				table[x index=0,y index=7] {fig/shkadov/ppo_1_jet.dat};
		\addplot [lower, forget plot] 				table[x index=0,y index=6] {fig/shkadov/ppo_1_jet.dat}; 
		\addplot [fill=blue3, opacity=0.5, forget plot] 	fill between[of=upper and lower];
		\addplot[draw=blue1, thick, smooth] 			table[x index=0,y index=5] {fig/shkadov/ppo_1_jet.dat}; 
			
		\addplot [upper, forget plot] 				table[x index=0,y index=7] {fig/shkadov/ppo_5_jets.dat};
		\addplot [lower, forget plot] 				table[x index=0,y index=6] {fig/shkadov/ppo_5_jets.dat}; 
		\addplot [fill=gray3, opacity=0.5, forget plot] 	fill between[of=upper and lower];
		\addplot[draw=gray1, thick, smooth] 		table[x index=0,y index=5] {fig/shkadov/ppo_5_jets.dat}; 

		\addplot [upper, forget plot] 				table[x index=0,y index=7] {fig/shkadov/ppo_10_jets.dat};
		\addplot [lower, forget plot] 				table[x index=0,y index=6] {fig/shkadov/ppo_10_jets.dat}; 
		\addplot [fill=green3, opacity=0.5, forget plot] 	fill between[of=upper and lower];
		\addplot[draw=green1, thick, smooth] 		table[x index=0,y index=5] {fig/shkadov/ppo_10_jets.dat}; 
			
	\end{axis}
\end{tikzpicture}
%%%%%%%%%%%%
\caption{\textbf{Score curves for different number of jets using the \textsc{ppo} algorithm to solve the \codeinline{shkadov-v0} environment.} For each curve, we plot the average (solid color) and the standard deviation (shaded color) obtained from $n_\text{training} = 5$ different runs. The dashed line indicates the reward obtained for the uncontrolled environment.} 
\label{fig:shkadov_score}
\end{figure} 
%%%%%%%%%%%%
%%%%%%%%%%%%
%%%
%%%%%%%%%%%%%
%%%%%%%%%%%%
\begin{figure}
\centering
%%%%%%%%%%%%
\pgfdeclarelayer{background}
\pgfsetlayers{background,main}
%%%%%%%%%%%%
\begin{tikzpicture}[	scale=0.8, trim axis left, trim axis right, font=\scriptsize]
	\begin{axis}[	xmin=0, xmax=25, ymin=-20, ymax=20, scale=1.0,
				width=\textwidth, height=.25\textwidth, scale only axis=true,
				legend cell align=left, legend pos=north east,
				grid=major, xlabel=$t$, ylabel=$x$]
				
		\legend{no control, \ppo}
				
		\addplot[draw=gray1, very thick, smooth] 	table[x index=0,y index=1] {fig/lorenz/lorenz_no_control.dat};
		\addplot[draw=green1, very thick, smooth] table[x index=0,y index=1] {fig/lorenz/lorenz_control.dat};
		
		\node[circle, fill=red, inner sep=0pt, minimum size=4pt] at (axis cs:3.45,18.5) {};
			
	\end{axis}
\end{tikzpicture}
%%%%%%%%%%%%
\caption{\textbf{Controlled versus uncontrolled time evolution of the $x$ parameter.} The red dot corresponds to the typical control peak that precedes the locking of the system, also observed in \cite{beintema2020}.} 
\label{fig:lorenz_control}
\end{figure} 
%%%%%%%%%%%%
%%%%%%%%%%%%
%
%%%%%%%%%%%%%
%%%%%%%%%%%%
\begin{figure}
\centering
%%%%%%%%%%%%
\pgfdeclarelayer{background}
\pgfsetlayers{background,main}
%%%%%%%%%%%%
%%%%%%%%%%%%

\begin{subfigure}[t]{.3\textwidth}
	\centering
	\fbox{\includegraphics[width=\textwidth]{fig/lorenz/t5.png}}
    	\caption{$t=5$}
	\label{fig:rayleigh_field_0}
\end{subfigure} \quad
\begin{subfigure}[t]{.3\textwidth}
	\centering
	\fbox{\includegraphics[width=\textwidth]{fig/lorenz/t25.png}}
    	\caption{$t=25$}
	\label{fig:rayleigh_field_2}
\end{subfigure} \quad
\begin{subfigure}[t]{.3\textwidth}
	\centering
	\fbox{\includegraphics[width=\textwidth]{fig/lorenz/t50.png}}
    	\caption{$t=50$}
	\label{fig:rayleigh_field_4}
\end{subfigure}

\medskip

\begin{subfigure}[t]{.3\textwidth}
	\centering
	\fbox{\includegraphics[width=\textwidth]{fig/lorenz/t100.png}}
    	\caption{$t=100$}
	\label{fig:rayleigh_field_8}
\end{subfigure} \quad
\begin{subfigure}[t]{.3\textwidth}
	\centering
	\fbox{\includegraphics[width=\textwidth]{fig/lorenz/t200.png}}
    	\caption{$t=200$}
	\label{fig:rayleigh_field_10}
\end{subfigure} \quad
\begin{subfigure}[t]{.3\textwidth}
	\centering
	\fbox{\includegraphics[width=\textwidth]{fig/lorenz/t300.png}}
    	\caption{$t=300$}
	\label{fig:rayleigh_field_12}
\end{subfigure}

%%%%%%%%%%%%
\caption{\textbf{Evolution of the controlled Lorenz system} using the \ppo algorithm.}
\label{fig:lorenz_fields}
\end{figure} 
%%%%%%%%%%%%
%%%%%%%%%%%%
