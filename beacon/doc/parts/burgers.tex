%%%%%%%%%%%%%%%%%%%%%%%%%%%%%%%%%%%%%%%%%%%%%%%%%%%%%%%%%%
%%%%%%%%%%%%%%%%%%%%%%%%%%%%%%%%%%%%%%%%%%%%%%%%%%%%%%%%%%
%%%%%%%%%%%%%%%%%%%%%%%%%%%%%%%%%%%%%%%%%%%%%%%%%%%%%%%%%%
%%%%%%%%%%%%%%%%%%%%%%%%%%%%%%%%%%%%%%%%%%%%%%%%%%%%%%%%%%
\chapter{Burgers}

%%%%%%%%%%%%%%%%%%%%%%%%%%%%%%%%%%%%%%%%%%%%%%%%%%%%%%%%%%
%%%%%%%%%%%%%%%%%%%%%%%%%%%%%%%%%%%%%%%%%%%%%%%%%%%%%%%%%%
%%%%%%%%%%%%%%%%%%%%%%%%%%%%%%%%%%%%%%%%%%%%%%%%%%%%%%%%%%
\section{Physics}

The inviscid Burgers equation was first introduced by Bateman in 1915, and models the behavior of a one-dimensional inviscid incompressible fluid flow \cite{bateman1915}, before being studied by Burgers in 1948 \cite{burgers1948}:

\begin{equation}
\label{eq:burgers}
	\dpart{u}{t} + u \dpart{u}{x} = 0.
\end{equation}

We consider the resolution of the Burgers equation on a domain of length $L$, along with the following initial and boundary conditions:

\begin{equation}
\label{eq:burgers_bc}
\begin{split}
	u(x,0)	&= u_\text{target}, \\
	u(0,t) 	&= u_\text{target} + \mathcal{U} ( -\sigma, \sigma), \\
	\dpart{u}{x} (L,t) &= 0,
\end{split}
\end{equation}

where $\sigma$ is the noise level introduced at the inlet, and $u_\text{target}$ is a constant value. The convection of the random inlet signal leads to a noisy solution in the domain, as is depicted in figure \ref{fig:burgers_free}. The initial perturbations steepen while propagating downstream to eventually form shocks.

%%%%%%%%%%%%
%%%%%%%%%%%%
\begin{figure}
\centering
%%%%%%%%%%%%
%%%%%%%%%%%%
\fbox{\includegraphics[width=.5\textwidth]{fig/burgers/free.png}}
%%%%%%%%%%%%
\caption{\textbf{Uncontrolled solution of the Burgers equation} with uniform random noise excitation at the inlet. The vertical bar indicates the position of the controller.}
\label{fig:burgers_free}
\end{figure} 
%%%%%%%%%%%%
%%%%%%%%%%%%

%%%%%%%%%%%%%%%%%%%%%%%%%%%%%%%%%%%%%%%%%%%%%%%%%%%%%%%%%%
%%%%%%%%%%%%%%%%%%%%%%%%%%%%%%%%%%%%%%%%%%%%%%%%%%%%%%%%%%
%%%%%%%%%%%%%%%%%%%%%%%%%%%%%%%%%%%%%%%%%%%%%%%%%%%%%%%%%%
\section{Discretization}

The Burgers equation (\ref{eq:burgers}) is discretized in time with a finite volume approach. The convective term is discretized using a TVD scheme with a Van Leer flux limiter, while the time marching is performed using a second-order finite-difference scheme.

%%%%%%%%%%%%%%%%%%%%%%%%%%%%%%%%%%%%%%%%%%%%%%%%%%%%%%%%%%
%%%%%%%%%%%%%%%%%%%%%%%%%%%%%%%%%%%%%%%%%%%%%%%%%%%%%%%%%%
%%%%%%%%%%%%%%%%%%%%%%%%%%%%%%%%%%%%%%%%%%%%%%%%%%%%%%%%%%
\section{Environment}

The goal of the environment is to control a pointwise forcing source term $a(t)$ on the right-hand side of (\ref{eq:burgers}), in order to damp the noise transported from the inlet. The forcing is applied at $x_p = 1$, while the length of the domain is set to $L=2$. The actions provided to the environment, which are expected to be in $\left[-1, 1\right]$, are then scale by an \textit{ad-hoc} non-dimensional amplitude factor $A=10$. The field is initially set equal to $u_\text{target} = 0.5$, and the variance of the inlet noise is chosen equal to $\sigma = 0.1$. The spatial discretization step is set to $\Delta x = \num{4e-3}$, while the numerical time-step is equal to $\Delta t = \num{8e43}$ time units. The action duration is set to $\Delta t_\text{act} = 0.05$ time units, for a total episode duration equal to $10$ time units, corresponding to $200$ actions. The observations provided to the agent are the $n_\text{obs} = 5$ values of $u$ upstream of the actuator. Finally, the reward is computed as:

\begin{equation}
\label{eq:burgers_reward}
	r(t) = - \Delta x \norm{u(x,t) - u_\text{target}}_2 \text{ with } x \in \left[ x_p, L \right].
\end{equation}

%%%%%%%%%%%%%%%%%%%%%%%%%%%%%%%%%%%%%%%%%%%%%%%%%%%%%%%%%%
%%%%%%%%%%%%%%%%%%%%%%%%%%%%%%%%%%%%%%%%%%%%%%%%%%%%%%%%%%
%%%%%%%%%%%%%%%%%%%%%%%%%%%%%%%%%%%%%%%%%%%%%%%%%%%%%%%%%%
\section{Results}

The environment as described in the previous section is referred to as \codeinline{burgers-v0}, in the fashion of the \textsc{gym} environments (the default parameters of the environment are provided in table \ref{table:burgers_parameters}). In this section, we provide some results related to its resolution using a \ppo agent (see the general hyperparameters in table \ref{table:default_ppo_parameters}). For the training, we set $n_\text{rollout} = 4000$, $n_\text{batch} = 2$, $n_\text{epoch} = 32$ and $n_\text{max} = 500k$.

The score curve is shown in figure \ref{fig:burgers_score}, while snapshots of the evolution of the controlled environment are shown in figure \ref{fig:burgers_fields}. As can be observed, the agent successfully damps the transported inlet noise following an opposition control strategy.

%%%%%%%%%%%%
%%%%%%%%%%%%
\begin{table}[h!]
    \footnotesize
    \caption{\textbf{Default parameters used for the \codeinline{burgers-v0} environment.}}
    \label{table:burgers_parameters}
    \centering
    \begin{tabular}{rll}
        \toprule
        	\codeinline{L}			& domain length		& $1$\\
        \codeinline{u_target}		& target value			& $0.5$\\
	\codeinline{sigma}		& inlet noise level		& $0.1$\\
	\codeinline{amp}		& control amplitude		& $10$\\
	\codeinline{ctrl_pos}		& control position		& $0.3$\\
        \bottomrule
    \end{tabular}
\end{table}
%%%%%%%%%%%%
%%%%%%%%%%%%

%%%%%%%%%%%%
%%%%%%%%%%%%
\begin{figure}
\centering
%%%%%%%%%%%%
\begin{tikzpicture}[	trim axis left, trim axis right, font=\scriptsize,
				upper/.style={	name path=upper, smooth, draw=none},
				lower/.style={	name path=lower, smooth, draw=none},]
	\begin{axis}[	xmin=0, xmax=500000, scale=0.75,
				ymin=-6, ymax=0,
				scaled x ticks=false,
				xtick={0, 100000, 200000, 300000, 400000, 500000},
				xticklabels={$0$,$100k$,$200k$,$300k$,$400k$,$500k$},
				legend cell align=left, legend pos=south east,
				legend style={nodes={scale=0.8, transform shape}},
				every tick label/.append style={font=\scriptsize},
				grid=major, xlabel=transitions, ylabel=score]
				
		\legend{no control, \ppo, \ddpg}
		
		\addplot[thick, opacity=0.7, dash pattern=on 2pt]	coordinates {(0,-3.5) (500000,-3.5)};
		
		\addplot [upper, forget plot] 				table[x index=0,y index=7] {fig/burgers/ppo.dat};
		\addplot [lower, forget plot] 				table[x index=0,y index=6] {fig/burgers/ppo.dat}; 
		\addplot [fill=blue3, opacity=0.5, forget plot] 	fill between[of=upper and lower];
		\addplot[draw=blue1, thick, smooth] 			table[x index=0,y index=5] {fig/burgers/ppo.dat}; 
		
		\addplot [upper, forget plot] 				table[x index=0,y index=7] {fig/burgers/ddpg.dat};
		\addplot [lower, forget plot] 				table[x index=0,y index=6] {fig/burgers/ddpg.dat}; 
		\addplot [fill=green3, opacity=0.5, forget plot] 	fill between[of=upper and lower];
		\addplot[draw=green1, thick, smooth] 		table[x index=0,y index=5] {fig/burgers/ddpg.dat}; 
			
	\end{axis}
\end{tikzpicture}
%%%%%%%%%%%%
\caption{\textbf{Score curves for the \codeinline{burgers-v0} environment} solved with \ppo and \ddpg. The dashed line indicates the reward obtained for the uncontrolled environment.} 
\label{fig:burgers_score}
\end{figure} 
%%%%%%%%%%%%
%%%%%%%%%%%%

%%%%%%%%%%%%
%%%%%%%%%%%%
\begin{figure}
\centering
%%%%%%%%%%%%
\pgfdeclarelayer{background}
\pgfsetlayers{background,main}
%%%%%%%%%%%%
%%%%%%%%%%%%

\begin{subfigure}[t]{\textwidth}
	\centering
	\begin{tikzpicture}[	scale=0.7, trim axis left, trim axis right, font=\scriptsize]
		\begin{axis}[	xmin=0, xmax=270, ymin=0, ymax=2, scale=1.0,
					xtick={0,50,100,150,200,250,300},
					width=\textwidth, height=.15\textwidth, scale only axis=true,
					legend cell align=left, legend pos=north east,
					grid=major, ylabel=$h$]
				
		\def\x{150}
		\def\w{2}
		\def\s{10}

		\draw[fill=green2, draw=gray2] 			(\x+0*\s-\w,1) rectangle (\x+0*\s+\w,1+0.87);
		\draw[fill=green2, draw=gray2] 			(\x+1*\s-\w,1) rectangle (\x+1*\s+\w,1-0.35);
		\draw[fill=green2, draw=gray2] 			(\x+2*\s-\w,1) rectangle (\x+2*\s+\w,1+0.29);
		\draw[fill=green2, draw=gray2] 			(\x+3*\s-\w,1) rectangle (\x+3*\s+\w,1-0.29);
		\draw[fill=green2, draw=gray2] 			(\x+4*\s-\w,1) rectangle (\x+4*\s+\w,1-0.74);
		\draw[fill=green2, draw=gray2] 			(\x+5*\s-\w,1) rectangle (\x+5*\s+\w,1-0.88);
		\draw[fill=green2, draw=gray2] 			(\x+6*\s-\w,1) rectangle (\x+6*\s+\w,1-0.57);
		\draw[fill=green2, draw=gray2] 			(\x+7*\s-\w,1) rectangle (\x+7*\s+\w,1+0.62);
		\draw[fill=green2, draw=gray2] 			(\x+8*\s-\w,1) rectangle (\x+8*\s+\w,1+0.47);
		\draw[fill=green2, draw=gray2] 			(\x+9*\s-\w,1) rectangle (\x+9*\s+\w,1-0.47);
		
		\addplot[draw=gray1, very thick, smooth] 	table[x index=0,y index=1] {fig/shkadov/field_200.dat};
			
		\end{axis}
	\end{tikzpicture}
    	\caption{$t=200$, start of control}
	\label{fig:shkadov_fields_200}
\end{subfigure}

\medskip

%%%%%%%%%%%%
\begin{subfigure}[t]{\textwidth}
	\centering
	\begin{tikzpicture}[	scale=0.7, trim axis left, trim axis right, font=\scriptsize]
		\begin{axis}[	xmin=0, xmax=270, ymin=0, ymax=2, scale=1.0,
					xtick={0,50,100,150,200,250,300},
					width=\textwidth, height=.15\textwidth, scale only axis=true,
					legend cell align=left, legend pos=north east,
					grid=major, ylabel=$h$]
				
			\def\x{150}
			\def\w{2}
			\def\s{10}
		
			\draw[fill=green2, draw=gray2] 			(\x+0*\s-\w,1) rectangle (\x+0*\s+\w,1+0.22);
			\draw[fill=green2, draw=gray2] 			(\x+1*\s-\w,1) rectangle (\x+1*\s+\w,1-0.13);
			\draw[fill=green2, draw=gray2] 			(\x+2*\s-\w,1) rectangle (\x+2*\s+\w,1-0.17);
			\draw[fill=green2, draw=gray2] 			(\x+3*\s-\w,1) rectangle (\x+3*\s+\w,1+0.077);
			\draw[fill=green2, draw=gray2] 			(\x+4*\s-\w,1) rectangle (\x+4*\s+\w,1+0.039);
			\draw[fill=green2, draw=gray2] 			(\x+5*\s-\w,1) rectangle (\x+5*\s+\w,1+0.045);
			\draw[fill=green2, draw=gray2] 			(\x+6*\s-\w,1) rectangle (\x+6*\s+\w,1-0.033);
			\draw[fill=green2, draw=gray2] 			(\x+7*\s-\w,1) rectangle (\x+7*\s+\w,1+0.095);
			\draw[fill=green2, draw=gray2] 			(\x+8*\s-\w,1) rectangle (\x+8*\s+\w,1-0.17);
			\draw[fill=green2, draw=gray2] 			(\x+9*\s-\w,1) rectangle (\x+9*\s+\w,1-0.12);
		
			\addplot[draw=gray1, very thick, smooth] 	table[x index=0,y index=1] {fig/shkadov/field_300.dat};
			
		\end{axis}
	\end{tikzpicture}
    	\caption{$t=300$}
	\label{fig:shkadov_fields_300}
\end{subfigure}

\medskip

%%%%%%%%%%%%
\begin{subfigure}[t]{\textwidth}
	\centering
	\begin{tikzpicture}[	scale=0.7, trim axis left, trim axis right, font=\scriptsize]
		\begin{axis}[	xmin=0, xmax=270, ymin=0, ymax=2, scale=1.0,
					xtick={0,50,100,150,200,250,300},
					width=\textwidth, height=.15\textwidth, scale only axis=true,
					legend cell align=left, legend pos=north east,
					grid=major, ylabel=$h$]
				
			\def\x{150}
			\def\w{2}
			\def\s{10}
		
			\draw[fill=green2, draw=gray2] 			(\x+0*\s-\w,1) rectangle (\x+0*\s+\w,1+0.12);
			\draw[fill=green2, draw=gray2] 			(\x+1*\s-\w,1) rectangle (\x+1*\s+\w,1+0.027);
			\draw[fill=green2, draw=gray2] 			(\x+2*\s-\w,1) rectangle (\x+2*\s+\w,1-0.22);
			\draw[fill=green2, draw=gray2] 			(\x+3*\s-\w,1) rectangle (\x+3*\s+\w,1+0.015);
			\draw[fill=green2, draw=gray2] 			(\x+4*\s-\w,1) rectangle (\x+4*\s+\w,1+0.065);
			\draw[fill=green2, draw=gray2] 			(\x+5*\s-\w,1) rectangle (\x+5*\s+\w,1-0.027);
			\draw[fill=green2, draw=gray2] 			(\x+6*\s-\w,1) rectangle (\x+6*\s+\w,1+0.0049);
			\draw[fill=green2, draw=gray2] 			(\x+7*\s-\w,1) rectangle (\x+7*\s+\w,1+0.0070);
			\draw[fill=green2, draw=gray2] 			(\x+8*\s-\w,1) rectangle (\x+8*\s+\w,1-0.12);
			\draw[fill=green2, draw=gray2] 			(\x+9*\s-\w,1) rectangle (\x+9*\s+\w,1-0.0091);
		
			\addplot[draw=gray1, very thick, smooth] 	table[x index=0,y index=1] {fig/shkadov/field_400.dat};
			
		\end{axis}
	\end{tikzpicture}
    	\caption{$t=400$}
	\label{fig:shkadov_fields_300}
\end{subfigure}

\medskip

%%%%%%%%%%%%
\begin{subfigure}[t]{\textwidth}
	\centering
	\begin{tikzpicture}[	scale=0.7, trim axis left, trim axis right, font=\scriptsize]
		\begin{axis}[	xmin=0, xmax=270, ymin=0, ymax=2, scale=1.0,
					xtick={0,50,100,150,200,250,300},
					width=\textwidth, height=.15\textwidth, scale only axis=true,
					legend cell align=left, legend pos=north east,
					grid=major, ylabel=$h$]
				
			\def\x{150}
			\def\w{2}
			\def\s{10}
		
			\draw[fill=green2, draw=gray2] 			(\x+0*\s-\w,1) rectangle (\x+0*\s+\w,1+0.098);
			\draw[fill=green2, draw=gray2] 			(\x+1*\s-\w,1) rectangle (\x+1*\s+\w,1+0.095);
			\draw[fill=green2, draw=gray2] 			(\x+2*\s-\w,1) rectangle (\x+2*\s+\w,1-0.18);
			\draw[fill=green2, draw=gray2] 			(\x+3*\s-\w,1) rectangle (\x+3*\s+\w,1-0.011);
			\draw[fill=green2, draw=gray2] 			(\x+4*\s-\w,1) rectangle (\x+4*\s+\w,1+0.083);
			\draw[fill=green2, draw=gray2] 			(\x+5*\s-\w,1) rectangle (\x+5*\s+\w,1-0.026);
			\draw[fill=green2, draw=gray2] 			(\x+6*\s-\w,1) rectangle (\x+6*\s+\w,1+0.010);
			\draw[fill=green2, draw=gray2] 			(\x+7*\s-\w,1) rectangle (\x+7*\s+\w,1+0.025);
			\draw[fill=green2, draw=gray2] 			(\x+8*\s-\w,1) rectangle (\x+8*\s+\w,1-0.12);
			\draw[fill=green2, draw=gray2] 			(\x+9*\s-\w,1) rectangle (\x+9*\s+\w,1-0.032);
		
			\addplot[draw=gray1, very thick, smooth] 	table[x index=0,y index=1] {fig/shkadov/field_500.dat};
			
		\end{axis}
	\end{tikzpicture}
    	\caption{$t=500$}
	\label{fig:shkadov_fields_500}
\end{subfigure}

\medskip

%%%%%%%%%%%%
\begin{subfigure}[t]{\textwidth}
	\centering
	\begin{tikzpicture}[	scale=0.7, trim axis left, trim axis right, font=\scriptsize]
		\begin{axis}[	xmin=0, xmax=270, ymin=0, ymax=2, scale=1.0,
					xtick={0,50,100,150,200,250,300},
					width=\textwidth, height=.15\textwidth, scale only axis=true,
					legend cell align=left, legend pos=north east,
					grid=major, ylabel=$h$]
				
			\def\x{150}
			\def\w{2}
			\def\s{10}
		
			\draw[fill=green2, draw=gray2] 			(\x+0*\s-\w,1) rectangle (\x+0*\s+\w,1+0.16);
			\draw[fill=green2, draw=gray2] 			(\x+1*\s-\w,1) rectangle (\x+1*\s+\w,1+0.042);
			\draw[fill=green2, draw=gray2] 			(\x+2*\s-\w,1) rectangle (\x+2*\s+\w,1-0.15);
			\draw[fill=green2, draw=gray2] 			(\x+3*\s-\w,1) rectangle (\x+3*\s+\w,1+0.031);
			\draw[fill=green2, draw=gray2] 			(\x+4*\s-\w,1) rectangle (\x+4*\s+\w,1+0.042);
			\draw[fill=green2, draw=gray2] 			(\x+5*\s-\w,1) rectangle (\x+5*\s+\w,1-0.058);
			\draw[fill=green2, draw=gray2] 			(\x+6*\s-\w,1) rectangle (\x+6*\s+\w,1-0.028);
			\draw[fill=green2, draw=gray2] 			(\x+7*\s-\w,1) rectangle (\x+7*\s+\w,1+0.058);
			\draw[fill=green2, draw=gray2] 			(\x+8*\s-\w,1) rectangle (\x+8*\s+\w,1-0.095);
			\draw[fill=green2, draw=gray2] 			(\x+9*\s-\w,1) rectangle (\x+9*\s+\w,1-0.014);
		
			\addplot[draw=gray1, very thick, smooth] 	table[x index=0,y index=1] {fig/shkadov/field_600.dat};
			
		\end{axis}
	\end{tikzpicture}
    	\caption{$t=600$}
	\label{fig:shkadov_fields_600}
\end{subfigure}

\medskip

%%%%%%%%%%%%
\begin{subfigure}[t]{\textwidth}
	\centering
	\begin{tikzpicture}[	scale=0.7, trim axis left, trim axis right, font=\scriptsize]
		\begin{axis}[	xmin=0, xmax=270, ymin=0, ymax=2, scale=1.0,
					xtick={0,50,100,150,200,250,300},
					width=\textwidth, height=.15\textwidth, scale only axis=true,
					legend cell align=left, legend pos=north east,
					grid=major, ylabel=$h$]
				
			\def\x{150}
			\def\w{2}
			\def\s{10}
		
			\draw[fill=green2, draw=gray2] 			(\x+0*\s-\w,1) rectangle (\x+0*\s+\w,1+0.21);
			\draw[fill=green2, draw=gray2] 			(\x+1*\s-\w,1) rectangle (\x+1*\s+\w,1+0.021);
			\draw[fill=green2, draw=gray2] 			(\x+2*\s-\w,1) rectangle (\x+2*\s+\w,1-0.15);
			\draw[fill=green2, draw=gray2] 			(\x+3*\s-\w,1) rectangle (\x+3*\s+\w,1+0.048);
			\draw[fill=green2, draw=gray2] 			(\x+4*\s-\w,1) rectangle (\x+4*\s+\w,1+0.044);
			\draw[fill=green2, draw=gray2] 			(\x+5*\s-\w,1) rectangle (\x+5*\s+\w,1-0.071);
			\draw[fill=green2, draw=gray2] 			(\x+6*\s-\w,1) rectangle (\x+6*\s+\w,1-0.038);
			\draw[fill=green2, draw=gray2] 			(\x+7*\s-\w,1) rectangle (\x+7*\s+\w,1+0.034);
			\draw[fill=green2, draw=gray2] 			(\x+8*\s-\w,1) rectangle (\x+8*\s+\w,1-0.093);
			\draw[fill=green2, draw=gray2] 			(\x+9*\s-\w,1) rectangle (\x+9*\s+\w,1-0.046);
		
			\addplot[draw=gray1, very thick, smooth] 	table[x index=0,y index=1] {fig/shkadov/field_700.dat};
			
		\end{axis}
	\end{tikzpicture}
    	\caption{$t=700$}
	\label{fig:shkadov_fields_700}
\end{subfigure}
%%%%%%%%%%%%
\caption{\textbf{Evolution of the flow under control of the agent, using 10 jets.} The jets strengths are represented with green rectangles.}
\label{fig:shkadov_fields}
\end{figure} 
%%%%%%%%%%%%
%%%%%%%%%%%%