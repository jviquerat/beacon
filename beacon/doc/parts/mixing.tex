%%%%%%%%%%%%%%%%%%%%%%%%%%%%%%%%%%%%%%%%%%%%%%%%%%%%%%%%%%
%%%%%%%%%%%%%%%%%%%%%%%%%%%%%%%%%%%%%%%%%%%%%%%%%%%%%%%%%%
%%%%%%%%%%%%%%%%%%%%%%%%%%%%%%%%%%%%%%%%%%%%%%%%%%%%%%%%%%
%%%%%%%%%%%%%%%%%%%%%%%%%%%%%%%%%%%%%%%%%%%%%%%%%%%%%%%%%%
\chapter{Mixing}

%%%%%%%%%%%%%%%%%%%%%%%%%%%%%%%%%%%%%%%%%%%%%%%%%%%%%%%%%%
%%%%%%%%%%%%%%%%%%%%%%%%%%%%%%%%%%%%%%%%%%%%%%%%%%%%%%%%%%
%%%%%%%%%%%%%%%%%%%%%%%%%%%%%%%%%%%%%%%%%%%%%%%%%%%%%%%%%%
\section{Physics}

We consider the resolution of the 2D Navier-Stokes equations coupled to a passive scalar convection-diffusion equation in a cavity of length $L$ and height $H$ with moving boundary conditions on all sides. The resulting system is driven by the following set of equations:

\begin{equation}
\label{eq:mixing}
\begin{split}
	\nabla \cdot \V{u} 						&= 0, \\
	\partial_t \V{u} + (\V{u} \cdot \nabla) \, \V{u} 	&= -\nabla p + \frac{1}{\re} \nabla^2 \V{u}, \\
	\partial_t c + (\V{u} \cdot \nabla) \, c 			&= \frac{1}{\pe} \nabla^2 c,
\end{split}
\end{equation}

where $\V{u}$ and $p$ are respectively the non-dimensional velocity and pressure of the fluid, and $c$ is the concentration of a passive species. The dynamics of the system (\ref{eq:mixing}) are controlled by two adimensional numbers. First, the Reynolds number $\re$, which represents the ratio between inertial and viscous forces:

\begin{equation*}
	\re = \frac{U L}{\nu},
\end{equation*}

where $U$ and $L$ are respectively the reference velocity and length values, and $\nu$ is the kinematic viscosity of the fluid. Second, the P\'eclet number $\pe$, which represents the ratio between the advective and diffusive transport rates:

\begin{equation*}
	\pe = \frac{U L}{D},
\end{equation*}

where $D$ is the diffusion coefficient of the considered species. In essence, a system with a high $\pe$ value presents a negligible diffusion, and scalar quantities move primarily due to fluid convection. The system (\ref{eq:mixing}) is completed by the following initial and boundary conditions:

\begin{equation}
\label{eq:mixing_bc}
\begin{split}
	\V{u}(x,y,0)	= 0 &\text{ and } c(x,y,0)	= c_0 \mathbb{1}_{\substack{ {x_\text{min} \leq x \leq x_\text{max}}\\ {y_\text{min} \leq y \leq y_\text{max}} }} \\
	\V{u}(x=0,y,t)	= (0, v_l) &\text{ and } \V{u}(x=L,y,t) = (0, v_r), \\
	\V{u}(x,y=0,t)	= (u_b, 0) &\text{ and } \V{u}(x,y=H,t) = (u_t, 0), \\
	\partial_y c(x,y=0,t) 	= 0 &\text{ and } \partial_y c(x,y=H,t) = 0, \\
	\partial_x c(x=0,y,t) = 0 &\text{ and } \partial_x c(x=L,y,t) = 0,
\end{split}
\end{equation}

where $\mathbb{1}$ is the indicator function, and $v_l$, $v_r$, $u_b$ and $u_t$ are user-defined values. In essence, the boundary conditions (\ref{eq:mixing_bc}) correspond to a multiple lid-driven cavity, where tangential velocity can be imposed independently on all sides, with an initial patch of concentration in the center of the domain.  Snapshots of the evolution of the system in time with $(v_l, v_r, u_b, u_t) = (0, 0, 1.0, -1.0)$ are presented in figure \ref{fig:mixing_example}.

%%%%%%%%%%%%
%%%%%%%%%%%%
\begin{figure}
\centering
%%%%%%%%%%%%
\pgfdeclarelayer{background}
\pgfsetlayers{background,main}
%%%%%%%%%%%%
%%%%%%%%%%%%

\begin{subfigure}[t]{.22\textwidth}
	\centering
	\fbox{\includegraphics[width=\textwidth]{fig/mixing/example_t0.png}}
    	\caption{$t=0$}
	\label{fig:mixing_example_0}
\end{subfigure} \quad
\begin{subfigure}[t]{.22\textwidth}
	\centering
	\fbox{\includegraphics[width=\textwidth]{fig/mixing/example_t10.png}}
    	\caption{$t=10$}
	\label{fig:mixing_example_10}
\end{subfigure} \quad
\begin{subfigure}[t]{.22\textwidth}
	\centering
	\fbox{\includegraphics[width=\textwidth]{fig/mixing/example_t30.png}}
    	\caption{$t=30$}
	\label{fig:mixing_example_30}
\end{subfigure} \quad
\begin{subfigure}[t]{.22\textwidth}
	\centering
	\fbox{\includegraphics[width=\textwidth]{fig/mixing/example_t60.png}}
    	\caption{$t=60$}
	\label{fig:mixing_example_60}
\end{subfigure}

%%%%%%%%%%%%
\caption{\textbf{Evolution of the concentration in time} with constant boundary conditions $(v_l, v_r, u_b, u_t) = (0, 0, 1.0, -1.0)$.}
\label{fig:mixing_example}
\end{figure} 
%%%%%%%%%%%%
%%%%%%%%%%%%

%%%%%%%%%%%%%%%%%%%%%%%%%%%%%%%%%%%%%%%%%%%%%%%%%%%%%%%%%%
%%%%%%%%%%%%%%%%%%%%%%%%%%%%%%%%%%%%%%%%%%%%%%%%%%%%%%%%%%
%%%%%%%%%%%%%%%%%%%%%%%%%%%%%%%%%%%%%%%%%%%%%%%%%%%%%%%%%%
\section{Discretization}
%
%The system (\ref{eq:rayleigh}) is discretized using a structured finite volume incremental projection scheme with centered fluxes, in the fashion of \cite{boivin2000}. For simplicity, the scheme is solved in a fully explicit way, except for the resolution of the Poisson equation for pressure. As is standard, a staggered grid is used for the finite volume scheme: the horizontal velocity is located on the west face of the cells, the vertical velocity is on the south face of the cells, while the pressure and temperature are located at the center of the cells. The computation of the instantaneous Nusselt number (\ref{eq:nusselt}) is performed by computing the first-order finite difference of the temperature between the center of the first cell at the bottom of the mesh and the reference temperature $T_H$. Doing so, we obtain $\nus = 2.16$  for $\ra = \num{1e4}$ once the permanent regime is reached, which is close to the reference values found in the literature \cite{ouertatani2008}.

%%%%%%%%%%%%%%%%%%%%%%%%%%%%%%%%%%%%%%%%%%%%%%%%%%%%%%%%%%
%%%%%%%%%%%%%%%%%%%%%%%%%%%%%%%%%%%%%%%%%%%%%%%%%%%%%%%%%%
%%%%%%%%%%%%%%%%%%%%%%%%%%%%%%%%%%%%%%%%%%%%%%%%%%%%%%%%%%
\section{Environment}

In the following, we set $H=1$, $L=1$, $(x_\text{min}, x_\text{max}, y_\text{min}, y_\text{max}) = (0.25, 0.75, 0.25, 0.75)$ and $c_0 = 2$ 
%The proposed environment is re-implemented based on the original work of Beintema \textit{et al.} \cite{beintema2020}. In the following, we set $\pr=0.71$, which corresponds to the parameter for air, and $\ra = \num{1e4}$ in order to avoid too high computational load. Similarly to \cite{beintema2020}, the control is performed by letting the DRL agent adjust the temperature of $n_s=10$ individual segments at the bottom of the cavity (see figure \ref{fig:rayleigh_sketch_env}). To do so, the actions proposed by the agent are continuous temperature fluctuations $\left\{ \hat{\theta}_i \right\}_{i \in \llbracket 0, n_s-1 \rrbracket}$ in the range $\left[-C, +C\right]$, with $C=0.75$. To enforce $\left< \theta(y=0,x,t) \right> = \theta_H$ and $\theta(y=0,x,t) \in \left[ \theta_H - C, \theta_H + C\right]$, the provided $\hat{\theta}_i$ are normalized as \cite{beintema2020}:
%
%\begin{equation}
%	\theta_i = \frac{\hat{\theta}_i - \left< \hat{\theta} \right>}{\max \left( 1, \max_j \left( \frac{\hat{\theta}_j - \left< \hat{\theta} \right>}{C} \right) \right)}.
%\end{equation}
%
%For simplicity, no interpolation is performed between actions, neither spatially nor temporally. The spatial discretization step is set as $\Delta x = 0.02$, while the numerical time step is $\Delta t = 0.01$. The action time-step $\Delta t_\text{act}$ is equal to $2$ time units, with the total episode length being fixed to $200$ time units, corresponding to $100$ actions.
%
%%%%%%%%%%%%%
%%%%%%%%%%%%
\begin{figure}
\centering
\def\sc{0.6}
%%%%%%%%%%%%
\begin{tikzpicture}[	scale=\sc,
				probe/.style={circle, fill=gray1, inner sep=0pt, minimum size=2.5pt}]

	%%% large rectangle
	\draw[fill=white, draw=black, very thick] (-0.01,-1) rectangle (10.01,10);

	%%% picture
	\node[anchor=south west,inner sep=0, scale=\sc] (image) at (0,0) {\includegraphics[width=9.99cm]{fig/rayleigh/T_no_control.png}};

	%%% probes
	\foreach \x in {0,...,7}
		\foreach \y in {0,...,7}
			\node[probe] at (0.5*1.25+1.25*\x,0.5*1.25+1.25*\y) {} ;
			
	%%% bottom frame
	\draw[fill=white] (0,-1) rectangle (10,0);
	\draw[dash pattern=on 2pt] (0,-0.5) -- (10, -0.5);
	\foreach \x in {1,...,9}
		\draw[dash pattern=on 1pt] (\x,-1) -- (\x,0);
		
	\draw[blue, thick] (0,-0.8) -- (1, -0.8);
	\draw[blue, thick] (1,-0.6) -- (2, -0.6);
	\draw[red,   thick] (2,-0.3) -- (3, -0.3);
	\draw[blue, thick] (3, -0.6) -- (4, -0.6);
	\draw[red, thick] (4,-0.2) -- (5, -0.2);
	\draw[red, thick] (5,-0.1) -- (6, -0.1);
	\draw[red, thick] (6,-0.3) -- (7, -0.3);
	\draw[blue, thick] (7,-0.7) -- (8, -0.7);
	\draw[blue, thick] (8,-0.9) -- (9, -0.9);
	\draw[blue, thick] (9,-0.6) -- (10, -0.6);

\end{tikzpicture}
%%%%%%%%%%%%
\caption{\textbf{Observation probes and actions imposition for the \codeinline{rayleigh-v0} environment}. The observations are collected at the probes regularly positioned in the domain, while the actions are imposed as piecewise-constant temperature boundary conditions on the bottom plate, with an average value equal to $\theta_H$.}
\label{fig:rayleigh_sketch_env}
\end{figure}
%%%%%%%%%%%%
%%%%%%%%%%%%
%
%The observations provided to the agent are the temperatures and their derivatives collected on a grid of $n_p \times n_p$ probes evenly spaced in the computational domain (see figure \ref{fig:rayleigh_sketch_env}). The resulting set of observations is flattened in a vector of size $2 \times (n_h+1)$.
%
%The reward at each time-step is simply set as the negative instantaneous Nusselt number, such that increasing the reward corresponds to a decrease of the temperature convection:
%
%\begin{equation}
%	r(t) = - \nus (t).
%\end{equation}
%
%Finally, each episode starts with the loading of a fully developed initial state obtained by solving the uncontrolled equations during a time $t_\text{init} = 200$ time units. The initial state corresponds to the field shown in figure \ref{fig:rayleigh_convection}. For convenience, this field is stored in a file and is loaded at the beginning of each episode. In case of modified parameters (spatial discretization, physical parameter...), this file can be generated by running the \codeinline{init.py} routine.
%
%%%%%%%%%%%%%%%%%%%%%%%%%%%%%%%%%%%%%%%%%%%%%%%%%%%%%%%%%%%
%%%%%%%%%%%%%%%%%%%%%%%%%%%%%%%%%%%%%%%%%%%%%%%%%%%%%%%%%%%
%%%%%%%%%%%%%%%%%%%%%%%%%%%%%%%%%%%%%%%%%%%%%%%%%%%%%%%%%%%
%\section{Results}
%
%The environment as described in the previous section is referred to as \codeinline{rayleigh-v0}, and its default parameters are provided in table \ref{table:rayleigh_parameters}. In this section, we provide some results related to its resolution using a \textsc{ppo} agent (see the general hyperparameters in table \ref{table:default_ppo_parameters}), with the specificity that the entropy bonus parameter is reduced to $\beta = 0.001$. For the training, we set $n_\text{rollout} = 1000$, $n_\text{batch} = 2$, $n_\text{epoch} = 32$ and $n_\text{max} = 600k$.
%
%The score curve obtained with the PPO algorithm is presented in \ref{fig:rayleigh_score}, while the time evolution of the Nusselt for the controlled versus uncontrolled cases are shown in figure \ref{fig:rayleigh_nusselt}. As can be observed, the agent manages to devise a set of transition actions toward a stationary state with $\nus (t) = 1$. The results of figure \ref{fig:rayleigh_nusselt} are in line with those of \cite{beintema2020}. In figure \ref{fig:rayleigh_fields}, we present the evolution of the temperature field during the first steps of the environment under the control of the agent using the default parameters.
%
%%%%%%%%%%%%
%%%%%%%%%%%%
%\begin{table}
%    \footnotesize
%    \caption{\textbf{Default parameters used for the \codeinline{rayleigh-v0} environment.}}
%    \label{table:rayleigh_parameters}
%    \centering
%    \begin{tabular}{rll}
%        \toprule
%        \codeinline{L}			& length of the domain					& $1$\\
%	\codeinline{H}			& height of the domain					& $1$\\
%	\codeinline{n_sgts}		& number of control segments				& $10$\\
%	\codeinline{ra}			& Rayleigh number						& $\num{1e4}$\\
%        \bottomrule
%    \end{tabular}
%\end{table}
%%%%%%%%%%%%
%%%%%%%%%%%%
%
%%%%%%%%%%%%%
%%%%%%%%%%%%
\begin{figure}
\centering
%%%%%%%%%%%%
\begin{tikzpicture}[	trim axis left, trim axis right, font=\scriptsize,
				upper/.style={	name path=upper, smooth, draw=none},
				lower/.style={	name path=lower, smooth, draw=none},]
	\begin{axis}[	xmin=0, xmax=200000, scale=0.75,
				ymin=-4, ymax=0,
				scaled x ticks=false,
				xtick={0, 50000, 100000, 150000, 200000},
				xticklabels={$0$,$50k$,$100k$,$150k$,$200k$},
				legend cell align=left, legend pos=south east,
				legend style={nodes={scale=0.8, transform shape}},
				every tick label/.append style={font=\scriptsize},
				grid=major, xlabel=transitions, ylabel=score]
				
		\legend{no control, 1 jet, 5 jets, 10 jets}
		
		\addplot[thick, opacity=0.7, dash pattern=on 2pt]	coordinates {(0,-2.90) (200000,-2.90)};
		
		\addplot [upper, forget plot] 				table[x index=0,y index=7] {fig/shkadov/ppo_1_jet.dat};
		\addplot [lower, forget plot] 				table[x index=0,y index=6] {fig/shkadov/ppo_1_jet.dat}; 
		\addplot [fill=blue3, opacity=0.5, forget plot] 	fill between[of=upper and lower];
		\addplot[draw=blue1, thick, smooth] 			table[x index=0,y index=5] {fig/shkadov/ppo_1_jet.dat}; 
			
		\addplot [upper, forget plot] 				table[x index=0,y index=7] {fig/shkadov/ppo_5_jets.dat};
		\addplot [lower, forget plot] 				table[x index=0,y index=6] {fig/shkadov/ppo_5_jets.dat}; 
		\addplot [fill=gray3, opacity=0.5, forget plot] 	fill between[of=upper and lower];
		\addplot[draw=gray1, thick, smooth] 		table[x index=0,y index=5] {fig/shkadov/ppo_5_jets.dat}; 

		\addplot [upper, forget plot] 				table[x index=0,y index=7] {fig/shkadov/ppo_10_jets.dat};
		\addplot [lower, forget plot] 				table[x index=0,y index=6] {fig/shkadov/ppo_10_jets.dat}; 
		\addplot [fill=green3, opacity=0.5, forget plot] 	fill between[of=upper and lower];
		\addplot[draw=green1, thick, smooth] 		table[x index=0,y index=5] {fig/shkadov/ppo_10_jets.dat}; 
			
	\end{axis}
\end{tikzpicture}
%%%%%%%%%%%%
\caption{\textbf{Score curves for different number of jets using the \textsc{ppo} algorithm to solve the \codeinline{shkadov-v0} environment.} For each curve, we plot the average (solid color) and the standard deviation (shaded color) obtained from $n_\text{training} = 5$ different runs. The dashed line indicates the reward obtained for the uncontrolled environment.} 
\label{fig:shkadov_score}
\end{figure} 
%%%%%%%%%%%%
%%%%%%%%%%%%
%
%%%%%%%%%%%%%
%%%%%%%%%%%%
\begin{figure}
\centering
%%%%%%%%%%%%
\begin{tikzpicture}[	trim axis left, trim axis right, font=\scriptsize,
				upper/.style={	name path=upper, smooth, draw=none},
				lower/.style={	name path=lower, smooth, draw=none},]
	\begin{axis}[	xmin=0, xmax=250, scale=0.75,
				ymin=0.8, ymax=2.5,
				scaled x ticks=false,
				legend cell align=left, legend pos=north east,
				legend style={nodes={scale=0.8, transform shape}},
				every tick label/.append style={font=\scriptsize},
				grid=major, xlabel=$t$, ylabel=$\nus(t)$]
				
		\legend{no control, \ppo}
		
		\addplot[thick, opacity=0.7, dash pattern=on 2pt]	coordinates {(0,2.16) (250,2.16)};
		
		\addplot[draw=blue1, thick, smooth, tension=0.05] 	table[x index=0,y index=1] {fig/rayleigh/nu.dat}; 
			
	\end{axis}
\end{tikzpicture}
%%%%%%%%%%%%
\caption{\textbf{Evolution of the instantaneous Nusselt number during an episode of the \codeinline{rayleigh-v0} environment} with and without control. The agent totally disables the convection, leading to a final Nusselt equal to $1$.} 
\label{fig:rayleigh_nusselt}
\end{figure} 
%%%%%%%%%%%%
%%%%%%%%%%%%
%
%%%%%%%%%%%%%
%%%%%%%%%%%%
\begin{figure}
\centering
%%%%%%%%%%%%
\pgfdeclarelayer{background}
\pgfsetlayers{background,main}
%%%%%%%%%%%%
%%%%%%%%%%%%

\begin{subfigure}[t]{.3\textwidth}
	\centering
	\fbox{\includegraphics[width=\textwidth]{fig/lorenz/t5.png}}
    	\caption{$t=5$}
	\label{fig:rayleigh_field_0}
\end{subfigure} \quad
\begin{subfigure}[t]{.3\textwidth}
	\centering
	\fbox{\includegraphics[width=\textwidth]{fig/lorenz/t25.png}}
    	\caption{$t=25$}
	\label{fig:rayleigh_field_2}
\end{subfigure} \quad
\begin{subfigure}[t]{.3\textwidth}
	\centering
	\fbox{\includegraphics[width=\textwidth]{fig/lorenz/t50.png}}
    	\caption{$t=50$}
	\label{fig:rayleigh_field_4}
\end{subfigure}

\medskip

\begin{subfigure}[t]{.3\textwidth}
	\centering
	\fbox{\includegraphics[width=\textwidth]{fig/lorenz/t100.png}}
    	\caption{$t=100$}
	\label{fig:rayleigh_field_8}
\end{subfigure} \quad
\begin{subfigure}[t]{.3\textwidth}
	\centering
	\fbox{\includegraphics[width=\textwidth]{fig/lorenz/t200.png}}
    	\caption{$t=200$}
	\label{fig:rayleigh_field_10}
\end{subfigure} \quad
\begin{subfigure}[t]{.3\textwidth}
	\centering
	\fbox{\includegraphics[width=\textwidth]{fig/lorenz/t300.png}}
    	\caption{$t=300$}
	\label{fig:rayleigh_field_12}
\end{subfigure}

%%%%%%%%%%%%
\caption{\textbf{Evolution of the controlled Lorenz system} using the \ppo algorithm.}
\label{fig:lorenz_fields}
\end{figure} 
%%%%%%%%%%%%
%%%%%%%%%%%%
