%%%%%%%%%%%%
%%%%%%%%%%%%
\begin{figure}
\centering
%%%%%%%%%%%%
\pgfdeclarelayer{background}
\pgfsetlayers{background,main}
%%%%%%%%%%%%
\begin{tikzpicture}[	scale=0.8, trim axis left, trim axis right, font=\scriptsize]
	\begin{axis}[	xmin=0, xmax=500, ymin=0.5, ymax=3.5, scale=1.0,
				width=\textwidth, height=.25\textwidth, scale only axis=true,
				legend cell align=left, legend pos=north east,
				grid=major, xlabel=$x$, ylabel=$h$]

		\begin{pgfonlayer}{background}
			\fill[color=bluegray3,opacity=0.3] (axis cs:0,0.5) rectangle (axis cs:150,5);
			\fill[color=orange3,opacity=0.3] (axis cs:150,0.5) rectangle (axis cs:275,5);
			\fill[color=teal3,opacity=0.3] (axis cs:275,0.5) rectangle (axis cs:500,5);
		\end{pgfonlayer}
		
		\addplot[draw=gray1, very thick, smooth] 			table[x index=0,y index=1] {fig/shkadov/shkadov_free.dat};
		\addplot[draw=black, thick, dash pattern=on 2pt] 	coordinates {(150,0.5) (150,5)}; 
		\addplot[draw=black, thick, dash pattern=on 2pt] 	coordinates {(275,0.5) (275,5)};
			
	\end{axis}
\end{tikzpicture}
%%%%%%%%%%%%
\caption{\textbf{Example of developed flow for the Shkadov equations with $\delta = 0.1$.} Three regions can be identified: a first region where the instability grows from a white noise (blue), a second region with pseudo-periodic waves (orange), and a third region with non-periodic, pulse-like waves (green).} 
\label{fig:shkadov_free}
\end{figure} 
%%%%%%%%%%%%
%%%%%%%%%%%%