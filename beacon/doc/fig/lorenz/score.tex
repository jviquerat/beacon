%%%%%%%%%%%%
%%%%%%%%%%%%
\begin{figure}
\centering
%%%%%%%%%%%%
\begin{tikzpicture}[	trim axis left, trim axis right, font=\scriptsize,
				upper/.style={	name path=upper, smooth, draw=none},
				lower/.style={	name path=lower, smooth, draw=none},]
	\begin{axis}[	xmin=0, xmax=2000000, scale=0.75,
				ymin=180, ymax=450,
				scaled x ticks=false,
				xtick={0, 500000, 1000000, 1500000, 2000000},
				xticklabels={$0$,$500k$,$1000k$,$1500k$,$2000k$},
				legend cell align=left, legend pos=north west,
				legend style={nodes={scale=0.8, transform shape}},
				every tick label/.append style={font=\scriptsize},
				grid=major, xlabel=transitions, ylabel=score]
				
		\legend{no control, \ppo}
		
		\addplot[thick, opacity=0.7, dash pattern=on 2pt]	coordinates {(0,207) (2000000,207)};
		
		\addplot [upper, forget plot] 				table[x index=0,y index=7] {fig/lorenz/ppo.dat};
		\addplot [lower, forget plot] 				table[x index=0,y index=6] {fig/lorenz/ppo.dat}; 
		\addplot [fill=blue3, opacity=0.5, forget plot] 	fill between[of=upper and lower];
		\addplot[draw=blue1, thick, smooth] 			table[x index=0,y index=5] {fig/lorenz/ppo.dat}; 
			
	\end{axis}
\end{tikzpicture}
%%%%%%%%%%%%
\caption{\textbf{Score curves using the \textsc{ppo} algorithm to solve the \codeinline{lorenz-v0} environment.} The dashed line indicates the reward obtained for the uncontrolled environement.} 
\label{fig:lorenz_score}
\end{figure} 
%%%%%%%%%%%%
%%%%%%%%%%%%