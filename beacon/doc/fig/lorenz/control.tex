%%%%%%%%%%%%
%%%%%%%%%%%%
\begin{figure}
\centering
%%%%%%%%%%%%
\pgfdeclarelayer{background}
\pgfsetlayers{background,main}
%%%%%%%%%%%%
\begin{tikzpicture}[	scale=0.8, trim axis left, trim axis right, font=\scriptsize]
	\begin{axis}[	xmin=0, xmax=25, ymin=-20, ymax=20, scale=1.0,
				width=\textwidth, height=.25\textwidth, scale only axis=true,
				legend cell align=left, legend pos=north east,
				grid=major, xlabel=$t$, ylabel=$x$]
				
		\legend{no control, \ppo}
				
		\addplot[draw=gray1, very thick, smooth] 	table[x index=0,y index=1] {fig/lorenz/lorenz_no_control.dat};
		\addplot[draw=green1, very thick, smooth] table[x index=0,y index=1] {fig/lorenz/lorenz_control.dat};
		
		\node[circle, fill=red, inner sep=0pt, minimum size=4pt] at (axis cs:3.45,18.5) {};
			
	\end{axis}
\end{tikzpicture}
%%%%%%%%%%%%
\caption{\textbf{Controlled (\ppo) versus uncontrolled time evolution of the $x$ parameter.} The red dot corresponds to the typical control peak that precedes the locking of the system, also observed in \cite{beintema2020}.} 
\label{fig:lorenz_control}
\end{figure} 
%%%%%%%%%%%%
%%%%%%%%%%%%