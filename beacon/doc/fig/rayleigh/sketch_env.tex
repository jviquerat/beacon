%%%%%%%%%%%%
%%%%%%%%%%%%
\begin{figure}
\centering
\def\sc{0.6}
%%%%%%%%%%%%
\begin{tikzpicture}[	scale=\sc,
				probe/.style={circle, fill=gray2, inner sep=0pt, minimum size=2.5pt}]

	%%% large rectangle
	\draw[fill=white, draw=black, very thick] (-0.01,-1) rectangle (10.01,10);

	%%% picture
	\node[anchor=south west,inner sep=0, scale=\sc] (image) at (0,0) {\includegraphics[width=9.99cm]{fig/rayleigh/T_control.png}};

	%%% probes
	\foreach \x in {0,...,3}
		\foreach \y in {0,...,3}
			\node[probe] at (0.5*2.5+2.5*\x,0.5*2.5+2.5*\y) {} ;
			
	%%% bottom frame
	\draw[fill=white] (0,-1) rectangle (10,0);
	\draw[dash pattern=on 2pt] (0,-0.5) -- (10, -0.5);
	\foreach \x in {1,...,9}
		\draw[dash pattern=on 1pt] (\x,-1) -- (\x,0);
		
	\draw[red, thick] (0,-0.5+0.5/0.75*0.60) -- (1, -0.5+0.5/0.75*0.60);
	\draw[blue, thick] (1,-0.5-0.5/0.75*0.54) -- (2, -0.5-0.5/0.75*0.54);
	\draw[red,   thick] (2,-0.5+0.5/0.75*0.29) -- (3, -0.5+0.5/0.75*0.29);
	\draw[blue, thick] (3, -0.5-0.5/0.75*0.53) -- (4, -0.5-0.5/0.75*0.53);
	\draw[red, thick] (4,-0.5+0.5/0.75*0.12) -- (5, -0.5+0.5/0.75*0.12);
	\draw[blue, thick] (5,-0.5-0.5/0.75*0.38) -- (6, -0.5-0.5/0.75*0.38);
	\draw[blue, thick] (6,-0.5-0.5/0.75*0.63) -- (7, -0.5-0.5/0.75*0.63);
	\draw[red, thick] (7,-0.5+0.5/0.75*0.47) -- (8, -0.5+0.5/0.75*0.47);
	\draw[blue, thick] (8,-0.5-0.5/0.75*0.34) -- (9, -0.5-0.5/0.75*0.34);
	\draw[red, thick] (9,-0.5+0.5/0.75*0.30) -- (10, -0.5+0.5/0.75*0.30);

\end{tikzpicture}
%%%%%%%%%%%%
\caption{\textbf{Observation probes and actions imposition for the \codeinline{rayleigh-v0} environment}. The observations are collected at the probes regularly positioned in the domain, while the actions are imposed as piecewise-constant temperature boundary conditions on the bottom plate, with an average value equal to $\theta_H$.}
\label{fig:rayleigh_sketch_env}
\end{figure}
%%%%%%%%%%%%
%%%%%%%%%%%%