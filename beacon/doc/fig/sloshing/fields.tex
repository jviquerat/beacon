%%%%%%%%%%%%
%%%%%%%%%%%%
\begin{figure}
\centering
%%%%%%%%%%%%

\begin{subfigure}[t]{\textwidth}
	\centering
	\fbox{\includegraphics[width=.4\textwidth]{fig/sloshing/control-0.png}} \hspace{.5cm} \fbox{\includegraphics[width=.4\textwidth]{fig/sloshing/no_control-0.png}}
    	\caption{$t=0$}
	\label{fig:sloshing_field_0}
\end{subfigure}

\medskip

\begin{subfigure}[t]{\textwidth}
	\centering
	\fbox{\includegraphics[width=.4\textwidth]{fig/sloshing/control-10.png}} \hspace{.5cm} \fbox{\includegraphics[width=.4\textwidth]{fig/sloshing/no_control-10.png}}
    	\caption{$t=0.5$}
	\label{fig:sloshing_field_10}
\end{subfigure} 

\medskip

\begin{subfigure}[t]{\textwidth}
	\centering
	\fbox{\includegraphics[width=.4\textwidth]{fig/sloshing/control-30.png}} \hspace{.5cm} \fbox{\includegraphics[width=.4\textwidth]{fig/sloshing/no_control-30.png}}
    	\caption{$t=1.5$}
	\label{fig:sloshing_field_30}
\end{subfigure} 

\medskip

\begin{subfigure}[t]{\textwidth}
	\centering
	\fbox{\includegraphics[width=.4\textwidth]{fig/sloshing/control-50.png}} \hspace{.5cm} \fbox{\includegraphics[width=.4\textwidth]{fig/sloshing/no_control-50.png}}
    	\caption{$t=2.5$}
	\label{fig:sloshing_field_50}
\end{subfigure} 

\medskip

\begin{subfigure}[t]{\textwidth}
	\centering
	\fbox{\includegraphics[width=.4\textwidth]{fig/sloshing/control-100.png}} \hspace{.5cm} \fbox{\includegraphics[width=.4\textwidth]{fig/sloshing/no_control-100.png}}
    	\caption{$t=5$}
	\label{fig:sloshing_field_100}
\end{subfigure} 

\medskip

\begin{subfigure}[t]{\textwidth}
	\centering
	\fbox{\includegraphics[width=.4\textwidth]{fig/sloshing/control-200.png}} \hspace{.5cm} \fbox{\includegraphics[width=.4\textwidth]{fig/sloshing/no_control-200.png}}
    	\caption{$t=10$}
	\label{fig:sloshing_field_10}
\end{subfigure} 

%%%%%%%%%%%%

%%%%%%%%%%%%
\caption{\textbf{Evolution of the fluid surface with (left) and without (right) agent control.} The control amplitude and direction is represented using the rectangle at the bottom (red means positive, blue means negative).}
\label{fig:sloshing_fields}
\end{figure} 
%%%%%%%%%%%%
%%%%%%%%%%%%