%%%%%%%%%%%%
%%%%%%%%%%%%
\begin{figure}
\centering
%%%%%%%%%%%%
\begin{tikzpicture}[]

	% ground
	\draw [thick] (-4,0) -- (4,0);

	% cart
	\draw [thick] (-2,0.5) -- (2,0.5);
	\draw [thick] (-2,0.5) -- (-2,2);
	\draw [thick] (2,0.5) -- (2,2);

	% wheels
	\draw [thick](-1.5,0.25) circle (0.25cm);
	\draw [thick] (1.5,0.25) circle (0.25cm);

	% water
	\begin{scope}
	    	\clip(-2,0.5) rectangle (2,2);
		\draw[draw=bluegray1,fill=cyan,thick,fill opacity=0.5] plot [smooth cycle] coordinates {(-2.0,0.5+0.8) (-1.0,0.5+1.2) (0.0,0.5+1.0) (1.0,0.5+0.5) (2.0,0.5+0.9) (2.0,0.5+0.0) (-2.0,0.5+0.0)};
	\end{scope}

	% arrows and stuff
	\draw[-stealth] (-1.0,0.5) -- (-1.0,1.7) node[pos=0.5, anchor=west] {$h$};
	\draw[] (0,0.5) -- (0,1.5);
	\draw[-stealth] (0,1.0) -- (0.5,1.0) node[pos=0.5, anchor=north] {$q$};
	\draw[-stealth] (2.0,1.0) -- (3.0,1.0) node[pos=0.5, anchor=south] {$\ddot{y}$};
	
\end{tikzpicture}
%%%%%%%%%%%%
\caption{\textbf{Configuration of the sloshing tank.} The fluid flow is determined by the fluid height $h(x,t)$ and by its mass flow rate $q(x,t)$. The movement of the tank is controlled by its acceleration $\ddot{y}(t)$.} 
\label{fig:sloshing_tank}
\end{figure} 
%%%%%%%%%%%%
%%%%%%%%%%%%