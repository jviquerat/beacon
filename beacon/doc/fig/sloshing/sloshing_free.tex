%%%%%%%%%%%%
%%%%%%%%%%%%
\begin{figure}
\centering
%%%%%%%%%%%%
%%%%%%%%%%%%
\begin{subfigure}{0.45\textwidth}
	\centering
	\begin{tikzpicture}[	scale=0.9, trim axis left, trim axis right, font=\scriptsize]
		\begin{axis}[	xmin=0, xmax=2.5, ymin=0, ymax=2, scale=1.0,
					xtick={0,0.5,1,1.5,2},
					width=\textwidth, height=.4\textwidth, scale only axis=true,
					legend cell align=left, legend pos=north east,
					grid=major, ylabel=$h$]
		
		\addplot[draw=bluegray1, very thick, smooth] 	table[x index=0,y index=1] {fig/sloshing/sloshing_excitation.dat};
			
		\end{axis}
	\end{tikzpicture}
    	\caption{Excitation phase}
	\label{fig:sloshing_excitation}
\end{subfigure} \quad
%%%%%%%%%%%%
\begin{subfigure}{0.45\textwidth}
	\centering
	\begin{tikzpicture}[	scale=0.9, trim axis left, trim axis right, font=\scriptsize]
		\begin{axis}[	xmin=0, xmax=2.5, ymin=0, ymax=2, scale=1.0,
					xtick={0,0.5,1,1.5,2},
					width=\textwidth, height=.4\textwidth, scale only axis=true,
					legend cell align=left, legend pos=north east,
					grid=major, ylabel={}]
		
			\addplot[draw=bluegray1, very thick, smooth] 	table[x index=0,y index=1] {fig/sloshing/sloshing_free.dat};
			
		\end{axis}
	\end{tikzpicture}
    	\caption{Relaxation phase}
	\label{fig:sloshing_free}
\end{subfigure}
%%%%%%%%%%%%
\caption{\textbf{Examples of fluid surface during the excitation phase (left) and the relaxation phase (right)}. The fluid height at rest is $h=1$.}
\label{fig:sloshing_examples}
\end{figure} 
%%%%%%%%%%%%
%%%%%%%%%%%%