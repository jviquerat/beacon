%%%%%%%%%%%%%%%%%%%%%%%% 
% Main packages
%%%%%%%%%%%%%%%%%%%%%%%%
\usepackage[a4paper]{geometry}						% manages document dimensions
\usepackage[utf8]{inputenc}							% translates various input encodings into latex internal language
\usepackage[T1]{fontenc}								% manages selection of font encoding
\usepackage[british]{babel}							% manages multilingual typesetting
\usepackage[osf]{libertine}							% libertine font
\usepackage[
	DIV=14,
	BCOR=2mm,
	headinclude=true,
	footinclude=false
	]{typearea}									% set page margins

%%%%%%%%%%%%%%%%%%%%%%%% 
% Math packages
%%%%%%%%%%%%%%%%%%%%%%%% 
\usepackage{amsmath}								% for main mathematical features
\usepackage{amsxtra}								% a few rarely used math-related commands
\usepackage{amssymb}								% extended symbol collection
\usepackage{amsthm}								% for theorem-like structures
\usepackage{ifthen}									% conditional commands
\usepackage{mathtools}								% extension of amsmath
\usepackage{stmaryrd} 								% symbols for computer science
\usepackage{siunitx} 								% formatting numbers with 10^
\usepackage{textcomp} 								% for many text symbols
\usepackage{bm} 									% boldmath symbols
\usepackage{esint} 									% for \oiint symbol and integrals style
\usepackage{fp}									% complex calculations

%%%%%%%%%%%%%%%%%%%%%%%% 
% Arrays packages
%%%%%%%%%%%%%%%%%%%%%%%% 
\usepackage{booktabs}								% enhances tables quality
\usepackage{multirow}								% create tab­u­lar cells span­ning mul­ti­ple rows
\usepackage{multicol}								% create tab­u­lar cells span­ning mul­ti­ple columns
\usepackage{bigdelim}								% multi-row delimiters in arrays
\usepackage{tabularx}								% extends tabular with adjustable-width columns 

%%%%%%%%%%%%%%%%%%%%%%%% 
% Layout packages
%%%%%%%%%%%%%%%%%%%%%%%% 
\usepackage{hyperref}								% production of hypertext links
\usepackage{enumitem}								% provides control over itemize and enumerate packages
\usepackage{ulem} 									% for underlining things
\usepackage{algorithm}								% typesetting for algorithms
\usepackage{algpseudocode}							% provides control over algorithm-defined structures
\usepackage{setspace}								% to set space between lines
\usepackage{lipsum}									% write random text
%\usepackage{listings}								% to typeset source code
\usepackage{bibentry}								% allows to put a full bib entry in the text of a document
\usepackage{pdflscape}								% to use landscape mode
\usepackage{fancyhdr}								% construction of headers and footers
\usepackage{ifoddpage}								% determine if current page is odd or even
\usepackage{lscape} 								% place selected parts of a document in landscape
\usepackage{epigraph}								% for epigraph typesetting
\setlength\epigraphwidth{8cm}							% set epigraph width
\setlength\epigraphrule{0pt}							% set width of the horizontal line appearing under the epigraph
\usepackage{titlesec}	 							% alternative section titles
\usepackage{color} 									% for custom chapter titles
\usepackage{textcomp} 								% To fix single quotes
\usepackage{xspace} 								% To handle trailing spaces

%%%%%%%%%%%%%%%%%%%%%%%% 
% Code typesetting packages
%%%%%%%%%%%%%%%%%%%%%%%% 
\usepackage{minted}									% for code typesetting
\usepackage{tcolorbox}								% for automatic boxing and framing
\usepackage{etoolbox}								% for automatic boxing and framing

%%%%%%%%%%%%%%%%%%%%%%%% 
% Math commands
%%%%%%%%%%%%%%%%%%%%%%%%
\newcommand{\R}{\mathbb{R}}									% real set
\newcommand{\N}{\mathbb{N}}									% natural set
\newcommand{\C}{\mathbb{C}}									% complex set
\newcommand\M{\mathbb{M}}									% mass matrix
\newcommand\K{\mathbb{K}}									% K matrix
\newcommand\I{\mathbb{I}}									% I matrix
\newcommand{\Nu}{\mathcal{V}}								% neighbor set for a cell
\newcommand{\inter}{\bigcap}									% intersection symbol
\newcommand{\union}{\bigcup}									% union symbol
\newcommand{\diff}{\text{d}}									% differential operator
\newcommand{\dpart}[2]{\frac{\partial #1}{\partial #2}}				% single partial differential operator
\newcommand{\ddpart}[2]{\frac{\partial^2 #1}{\partial {#2}^2}}			% double partial differential operator
\newcommand{\rot}{\nabla \times}								% curl operator
\renewcommand{\div}{\nabla \cdot}								% divergence operator
\newcommand{\module}[1]{\mathopen| #1 \mathclose|}				% modulus operator
\newcommand{\norm}[1]{\left\| #1 \right\|}							% general norm operator
\newcommand{\eps}{\varepsilon}								% custom epsilon sign
\newcommand{\e}{\, \mathrm{e}}								% short exponential notation
\newcommand{\V}[1]{\bm{#1}}								% vector notation
\newcommand{\GV}[1]{\boldsymbol{#1}}							% vector notation for complex cases
\newcommand{\Dt}{\Delta t}									% timestep notation

%%%%%%%%%%%%%%%%%%%%%%%% 
% Figures packages
%%%%%%%%%%%%%%%%%%%%%%%% 
\usepackage{graphicx}								% graphics support
\usepackage[font={footnotesize},labelfont={bf}]{caption} 		% style of figure captions
\usepackage{fp}									% high-precision arithmetic operations
\usepackage[final]{pdfpages} 							% inclusion of PDF pages (must be put BEFORE pgfplots, tikz and externalize stuff)
\usepackage{subcaption}								% subfigure support
\usepackage{animate}								% to create animations
\usepackage{pgfplots}								% pgfplots support
\usepackage{pgfplotstable}							% rounding and formatting of numerical tables for pgfplots
\usepgfplotslibrary{fillbetween}							% filling between curves
\usepgfplotslibrary{polar}								% polar plots
\usepgfplotslibrary{patchplots}							% plot triangular or quadrangular patches on graphs
\usepackage{tikz}									% tikz support
\usepackage{tikz-3dplot}								% 3D plots
\usetikzlibrary{calc}									% complex coordinates calculations
\usetikzlibrary{fadings}								% to apply fadings
\usetikzlibrary{shapes.misc}							% to define various complex shapes
\usetikzlibrary{shapes.geometric}						% to define geometric shapes such as polygons or ellipses 
\usetikzlibrary{3d}									% drawing 3d shapes
\usetikzlibrary{arrows}								% arrow tips
\usetikzlibrary{decorations}							% basic decoration stuff
\usetikzlibrary{decorations.pathmorphing}					% deformation of to-be-decorated paths
\usetikzlibrary{decorations.text}							% decoration of paths with text
\usetikzlibrary{decorations.markings}						% putting marks on paths
\usetikzlibrary{intersections}							% automated calculations of paths intersections
\usetikzlibrary{patterns}								% defines patterns for filling areas
\usetikzlibrary{spy}									% to create local close-ups on plots
\usetikzlibrary{matrix}								% to draw matrices
\usetikzlibrary{chains}								% to create chains (timeline for example)
\usetikzlibrary{positioning}								% additional options for placing nodes
\usetikzlibrary{shapes.multipart}						% to generate multi-parts nodes
\usetikzlibrary{fit}									% to draw encircling paths
\pgfplotsset{compat=1.12}								% require a specific version of pgfplots

%%%%%%%%%%%%%%%%%%%%%%%% 
% Handle tikzExternalize
% If ever problems appear with tikzexternalize, you should check the preferences of TexShop
% Go to "Moteur", and add "--shell-escape" to the line pdfTex/Latex, then re-run
% The prefix file must exist before the compilation (fig/ or ext/ or anything you choose)
%%%%%%%%%%%%%%%%%%%%%%%% 
\usepackage{etex} 									% extends memory
\usepgfplotslibrary{external} 							% converts each picture as a separate pdf
\tikzexternalize[optimize command away=\includepdf] 		% the option is for includepdf compatibility (from pdfpages package)
\tikzsetexternalprefix{ext/}								% put externally compiled figures in ext/	

%%%%%%%%%%%%%%%%%%%%%%%% 
% Color packages and definitions
%%%%%%%%%%%%%%%%%%%%%%%% 
\usepackage{xcolor}									% driver-independant colors extensions
\usepackage{color} 									% foreground and background color control 

\definecolor{blue1}		{RGB}{0,177,234}				% deep cyan
\definecolor{blue2}		{RGB}{76,200,239}				% lighter cyan
\definecolor{blue3}		{RGB}{127,215,244}				% lighter cyan
\definecolor{blue4}		{RGB}{178,231,248}				% lighter cyan
\definecolor{bluegray1}	{RGB}{0,127,167}				% deep cyan/gray mix
\definecolor{bluegray2}	{RGB}{76,165,193}				% lighter cyan/gray mix
\definecolor{bluegray3}	{RGB}{127,191,211}				% lighter cyan/gray mix
\definecolor{bluegray4}	{RGB}{178,216,228}				% lighter cyan/gray mix
\definecolor{gray1}		{RGB}{76,84,93}				% deep gray
\definecolor{gray2}		{RGB}{129,135,141}				% lighter gray
\definecolor{gray3}		{RGB}{165,169,174}				% lighter gray
\definecolor{gray4}		{RGB}{201,203,206}				% lighter gray
\definecolor{gray5}		{RGB}{230,230,230}				% lighter gray
\definecolor{gray6}		{RGB}{245,245,245}				% lighter gray
\definecolor{orange1}	{RGB}{255,126,46}				% deep orange
\definecolor{orange2}	{RGB}{255,164,108}				% lighter orange
\definecolor{orange3}	{RGB}{255,190,150}				% lighter orange
\definecolor{orange4}	{RGB}{255,216,192}				% lighter orange
\definecolor{brown1}		{RGB}{205,133,63}				% brown
\definecolor{green1}		{RGB}{0,168,107}				% green
\definecolor{green2}		{RGB}{30,198,137}				% green
\definecolor{green3}		{RGB}{60,228,167}				% green
\definecolor{green4}		{RGB}{90,255,197}				% green
\definecolor{purple1}		{RGB}{89,89,171}				% deep purple
\definecolor{purple2}		{RGB}{159,159,195}				% lighter purple
\definecolor{purple3}		{RGB}{189,189,225}				% lighter purple
\definecolor{purple4}		{RGB}{209,209,255}				% lighter purple
\definecolor{teal1}		{RGB}{0,128,128}				% deep teal
\definecolor{teal2}		{RGB}{100,168,168}				% deep teal
\definecolor{teal3}		{RGB}{130,198,198}				% deep teal
\definecolor{teal4}		{RGB}{160,228,228}				% deep teal
\definecolor{red1}		{RGB}{255,0,0}					% deep red
\definecolor{red2}		{RGB}{255,30,30}				% deep red
\definecolor{red3}		{RGB}{255,60,60}				% deep red
\definecolor{red4}		{RGB}{255,90,90}				% deep red
\definecolor{royal1}		{RGB}{2,119,189}				% royal blue
\definecolor{royal2}		{RGB}{32,149,219}				% royal blue
\definecolor{royal3}		{RGB}{62,179,249}				% royal blue
\definecolor{royal4}		{RGB}{92,209,255}				% royal blue

\def\smallopacity	{0.7}								% slightly opaque
\def\bigopacity		{0.5}								% more opaque
\def\verybigopacity	{0.3}								% very opaque

%%%%%%%%%%%%%%%%%%%%%%%% 
% TikZ customizations
%%%%%%%%%%%%%%%%%%%%%%%%
% Figure and tikz sizes
\newcommand{\tikzscaleforoneplot}		{1.0}						% figure scale for a single tikz plot
\newcommand{\tikzscalefortwoplots}		{0.8}						% figure scale for two tikz plots side by side
\newcommand{\tikzscaleforthreeplots}	{0.675}					% figure scale for three tikz plots side by side
\newcommand{\tikzwidthforoneplot}		{.65\textwidth}				% figure width for a single tikz plots
\newcommand{\tikzheightforoneplot}		{.4\textwidth}				% figure height for a single tikz plots
\tikzset{every mark/.append style={scale=0.4}}						% size of tikz marks
\tikzset{font=\small}											% size of fonts in tikz

\pgfplotsset{legend style={font=\footnotesize}}						% size of legend text
\pgfplotsset{every tick label/.append style={font=\footnotesize}}			% size of ticks in plots

\newcommand{\scaleforzeroplot}{0.9\textwidth}						% special figure scale
\newcommand{\scaleforoneplot}{0.67\textwidth}					% figure scale for one plot
\newcommand{\scalefortwoplots}{.45\textwidth}						% figure scale for two plots side by side
\newcommand{\scaleforthreeplots}{.28\textwidth}					% figure scale for three plots side by side
\newcommand{\exceptionalscaleforthreeplots}{.22\textwidth}			% special figure scale

% Tikz styles for arrays and matrices
\tikzset{
	mStyle/.style={row sep=-2*\pgflinewidth, column sep=-2*\pgflinewidth},
	mWH/.style n args={2}{minimum width=#1cm, minimum height=#2cm},
	mNode/.style n args={1}{draw=blue1, thick, fill=#1, mWH},
	mRounded/.style n args={5}{append after command={\pgfextra
        		\draw[mNode={#5}, sharp corners]% 
    		(\tikzlastnode.west)% 
    		[rounded corners=#1pt] |- (\tikzlastnode.north)% 
    		[rounded corners=#2pt] -| (\tikzlastnode.east)% 
    		[rounded corners=#3pt] |- (\tikzlastnode.south)% 
    		[rounded corners=#4pt] -| (\tikzlastnode.west);%
   		\endpgfextra}},
	mTL/.style n args={1}{mRounded={3}{0}{0}{0}{#1}},
	mTR/.style n args={1}{mRounded={0}{3}{0}{0}{#1}},
	mBL/.style n args={1}{mRounded={0}{0}{0}{3}{#1}},
	mBR/.style n args={1}{mRounded={0}{0}{3}{0}{#1}},
	mL/.style n args={1}{mRounded={3}{0}{0}{3}{#1}},
	mR/.style n args={1}{mRounded={0}{3}{3}{0}{#1}},
	mN/.style n args={1}{mRounded={0}{0}{0}{0}{#1}},
	mB/.style n args={1}{mRounded={0}{0}{3}{3}{#1}},
	mT/.style n args={1}{mRounded={3}{3}{0}{0}{#1}}
}

% Tikz style for mesh-related drawings
\tikzset{
	vertex/.style={draw, thick, shape=circle, inner sep=2pt, fill=blue1},
	edge/.style={draw, thick, gray2}
}

% Default arrow style
\tikzset{arrow/.style={thick,gray2,-stealth}}

% Default label style for arrays
\tikzset{label/.style={text depth=.3\baselineskip, text height=.5\baselineskip}}

% Tikz style for classes
\tikzset{
	cpp_class/.style={rectangle, rounded corners, text width=2cm, text centered, draw=blue1, ultra thick, fill=gray6}
}

% Style of fbox border, usually used to frame jpg/png figures
\setlength{\fboxsep}{0pt}%
\setlength{\fboxrule}{1pt}%

%%%%%%%%%%%%%%%%%%%%%%%% 
% minted settings, for code typesetting
% minted requires several things:
% - the Pygments package must be installed -> pip install Pygments
% - the --shell-escape must be added to the pdflatex command line
%%%%%%%%%%%%%%%%%%%%%%%% 
\usemintedstyle{manni}
\setminted{	fontsize=\scriptsize,
			fontfamily=courier,
			fontseries=b,
			obeytabs=true,
			tabsize=2}
\setmintedinline{fontsize=\footnotesize}
\definecolor{mintedbg}{RGB}{200,200,200}
\newtcolorbox{mintedbox}{	colframe=mintedbg, 
						colback=mintedbg, 
						coltitle=mintedbg}
\BeforeBeginEnvironment{minted}{\begin{mintedbox}}
\AfterEndEnvironment{minted}{\end{mintedbox}}
\newcommand{\codeinline}[1]{\mintinline{cpp}|#1|}

%%%%%%%%%%%%%%%%%%%%%%%% 
% Book customizations
%%%%%%%%%%%%%%%%%%%%%%%%
% Chapter title style
\titleformat{\chapter}[display] %
{\bfseries\Large} %
{ %
\hfill\fontsize{100}{70}\selectfont\color{lightgray}\textsc{\thechapter}} %
{0ex} %
{\filleft\fontsize{40}{45}\selectfont\scshape} %
[\vspace{0ex}] %

% Custom choice for "veuves et orphelines"
\widowpenalty=9999
\clubpenalty=9999

% No indentation when starting paragraphs
\setlength\parindent{0pt}

% Url style
\urlstyle{rm}

% Hyperlinks style
\hypersetup{
    colorlinks 	= true,										% allow link coloring
    breaklinks 	= true,										% allow linebreaks for long links
    urlcolor 		= purple,										% url color
    linkcolor 	= gray1,									% internal links color
    citecolor 	= bluegray1,									% bibliographic link color
    pdfborder 	= {0 0 0}										% border style
}

% Custom caption fonts
\DeclareCaptionLabelSeparator*{vbar}{{\bfseries{} | }}
\captionsetup{labelsep=vbar}

% Custom enumerate style
\setenumerate{label=\textcolor{bluegray1}{$\bm{\diamond}$},itemsep=1pt,topsep=5pt} 

% Custom displaystyle array
\newenvironment{disarray}{
	\everymath{\displaystyle\everymath{}}\array}
{\endarray}

% Redefine \emph and \complete
\renewcommand{\emph}[1]{\textbf{\textcolor{teal1}{#1}}}
\newcommand{\warning}[1]{\textbf{\textcolor{red}{#1}}}
  
% A remark environment
\newenvironment{remark}{\medskip\itshape\noindent\small \textbf{Remark}:}{\medskip}

% Define multicols separation
\setlength\columnsep{20pt}

% Usual textual things
\newcommand{\eg}{\textit{e.g.}\;}								% e.g. notation
\newcommand{\ie}{\textit{i.e.}\;}									% i.e. notation
\newcommand{\etal}{\textit{et al.}\;}								% et. al. notation

% Algorithms
\newcommand{\vpg}{\textsc{vpg}\xspace}
\newcommand{\atc}{\textsc{a2c}\xspace}
\newcommand{\ppo}{\textsc{ppo}\xspace}
\newcommand{\dqn}{\textsc{dqn}\xspace}
\newcommand{\td}{\textsc{td3}\xspace}
\newcommand{\sac}{\textsc{sac}\xspace}
\newcommand{\ddpg}{\textsc{ddpg}\xspace}

\newcommand{\gym}{\textsc{gym}\xspace}

% Fancy page style for books
\fancypagestyle{plain}{
	\fancyhf{}
	\fancyfoot[LE]{\footnotesize \sffamily \bfseries \thepage}
	\fancyfoot[RO]{\footnotesize \sffamily \bfseries \thepage}
	\renewcommand{\headrulewidth}{0pt}
	\renewcommand{\footrulewidth}{0pt}
}

\pagestyle{fancy}
\fancyhf{}
\fancyhead[LE]{\footnotesize \sffamily \bfseries }
\fancyhead[RO]{\footnotesize \sffamily \bfseries }
\fancyfoot[LE]{\footnotesize \sffamily \bfseries \thepage}
\fancyfoot[RO]{\footnotesize \sffamily \bfseries \thepage}
	
\makeatletter
\def\parsenode[#1]#2\pgf@nil{%
   	\tikzset{label node/.style={#1}}
    	\def\nodetext{#2}
}

% Page insertion and jump
\newcommand{\insertemptypage}{								%
	\clearpage \thispagestyle{empty} \strut \newpage}				% insert an empty page	
\newcommand{\jumpandstartoneven}{							%
	\clearpage \checkoddpage \ifoddpage \insertemptypage \else \fi}	% go to next even page
\newcommand{\jumpandstartonodd}{								%
	\clearpage \checkoddpage \ifoddpage \else \insertemptypage \fi}	% go to next odd page
	
%%%%%%%%%%%%%%%%%%%%%%%% 
% Other customizations
%%%%%%%%%%%%%%%%%%%%%%%% 
\usepackage{nicefrac}
\newcommand\bsfrac[2]{\reflectbox{\scriptsize\nicefrac[\reflectbox]{$#1$}{$#2$}}}